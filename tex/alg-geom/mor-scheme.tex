\chapter{Morphisms of locally ringed spaces}
Having set up the definition of a locally ringed space,
we are almost ready to define morphisms between them.
Throughout this chapter, you should imagine your ringed spaces
are the affine schemes we have so painstakingly defined;
but it will not change anything to work in the generality
of arbitrary locally ringed spaces.

\section{Morphisms of ringed spaces via sections}
Let $(X, \OO_X)$ and $(Y, \OO_Y)$ be ringed spaces.
We want to give a define what it means
to have a function $\pi \colon X \to Y$ between them.\footnote{Notational
	connotations: for ringed spaces, $\pi$ will be used for maps,
	since $f$ is often used for sections.}
We start by requiring the map to be continuous,
but this is not enough: there is a sheaf on it!

Well, you might remember what we did for baby ringed spaces:
any time we had a function on an open set of $U \subseteq Y$,
we wanted there to be an analogous function on $\pi\pre(U) \subseteq X$.
For baby ringed spaces, this was done by composition,
since the elements of the sheaf \emph{were} really complex valued functions:
\[ \pi^\sharp\phi \text{ was defined as } \phi \circ \pi. \]
The upshot was that we got a map $\OO_Y(U) \to \OO_X(\pi\pre(U))$
for every open set $U$.
\begin{center}
	\begin{asy}
		size(13cm);
		bigblob("$X$");
		pair p = (0.5,0);
		filldraw(CR(p, 1), opacity(0.2)+lightgreen, deepgreen+dashed);
		label("$\pi^{\text{pre}}(U)$", p+dir(135), dir(135), deepgreen);

		transform t = scale(0.8) * shift(14*dir(180));
		add(t * CC());
		p = t*p;

		bigblob("$Y$");
		pair q = (0,0.5);
		filldraw(CR(q, 1.2), opacity(0.2)+lightred, red+dashed);
		label("$U$", q+1.2*dir(45), dir(45), red);

		draw( (-9,0.5)--(-3,0.5), EndArrow);
		label("$\pi$", (-6,0.5), dir(90));
		label("$\boxed{f \in \mathcal O_Y(U)}$", (0,-5), red);
		label("$\boxed{\pi^\sharp f \in \mathcal O_X(f^{\text{pre}}(U))}$",
			t*(0,-6), deepgreen);
	\end{asy}
\end{center}

Now, for general locally ringed spaces,
the sections are just random rings,
which may not be so well-behaved.
So the solution is that we \emph{include} the data of $f^\sharp$
as part of the definition of a morphism.
\begin{definition}
	A \vocab{morphism of ringed spaces}
	$(X, \OO_X) \to (Y, \OO_Y)$ consists of a pair $(\pi, \pi^\sharp)$
	where $\pi \colon X \to Y$ is a continuous map
	(of topological spaces),
	and $\pi^\sharp$ consists of a choice of ring homomorphism
	\[ \pi^\sharp_U \colon \OO_Y(U) \to \OO_X(\pi\pre(U)) \]
	for every open set $U \subseteq Y$, such that the restriction diagram
	\begin{center}
	\begin{tikzcd}
		\OO_Y(U) \ar[r] \ar[d] & \OO_Y(\pi\pre(U)) \ar[d] \\
		\OO_Y(V) \ar[r] & \OO_X(\pi\pre(V))
	\end{tikzcd}
	\end{center}
	commutes for $V \subseteq U$.
\end{definition}
\begin{abuse}
	We will abbreviate $(\pi, \pi^\sharp) \colon (X, \OO_X) \to (Y, \OO_Y)$
	to just $\pi \colon X \to Y$,
	despite the fact this notation is exactly
	the same as that for topological spaces.
\end{abuse}

There is an obvious identity map,
and so we can also define isomorphism etc.\ in the categorical way.

\section{Morphisms of ringed spaces via stalks}
Unsurprisingly, the sections are clumsier to work with
than the stalks, now that we have grown to love localization.
So rather than specifying $\pi^\sharp_U$ on every open set $U$,
it seems better if we could do it by stalks
(there are fewer stalks than open sets,
so this saves us a lot of work!).

We start out by observing that we \emph{do}
get a morphism of stalks.
\begin{proposition}
	[Induced stalk morphisms]
	If $\pi \colon X \to Y$ is a map of ringed spaces
	sending $\pi(p) = q$, then we get a map
	\[ \pi_p^\sharp \colon \OO_{Y,q} \to \OO_{X,p} \]
	whenever $\pi(p) = q$.
\end{proposition}

This means you can draw a morphism of locally ringed spaces
as a continuous map on the topological space,
plus for each $\pi(p) = q$,
an assignment of each germ at $q$ to a germ at $p$.
\begin{center}
\begin{asy}
	size(10cm);
	pair p = (-4,0);
	pair q = ( 4,0);
	filldraw(ellipse(p, 2, 1 ), opacity(0.2)+lightcyan, black);
	filldraw(ellipse(q, 2, 1 ), opacity(0.2)+lightcyan, black);
	label("$X$", p+dir(270), dir(270), blue);
	label("$Y$", q+dir(270), dir(270), blue);
	pair pt = (-4, 3.5);
	pair qt = ( 4, 4.3);
	draw( p--pt, red );
	draw( q--qt, red );
	label("$\mathcal O_{X,p}$", pt, dir(90), red);
	label("$\mathcal O_{Y,q}$", qt, dir(90), red);
	dot("$p$", p, dir(-90), blue);
	dot("$q$", q, dir(-90), blue);
	pair pg = (-4, 1.6);
	pair qg = (4, 2.8);
	dot("$[s]_q$", qg, dir(0), red+6);
	dot("$[\pi^\sharp(s)]_p$", pg, dir(180), red+6);
	draw(qg--pg, deepgreen, EndArrow, Margin(4,4));
	label("$\pi^\sharp_p$", midpoint(qg--pg), dir(90), deepgreen);
	draw( (-1.5,0)--(1.5,0), deepgreen, EndArrow );
	label("$\pi$", origin, dir(-90), deepgreen);
\end{asy}
\end{center}
Again, compare this to the pullback picture:
this is roughly saying that if a function $f$ has some enriched value at $q$,
then $\pi^\sharp(f)$ should be assigned a corresponding enriched value at $p$.
The analogy is not perfect since the stalks at $q$ and $p$
may be different rings in general,
but there should at least be a ring homomorphism (the assignment).
\begin{proof}
	If $(s,U)$ is a germ at $q$,
	then $(\pi^\sharp(s), \pi\pre(U))$ is a germ at $p$,
	and this is a well-defined morphism because
	of compatibility with restrictions.
\end{proof}

We already obviously have uniqueness in the following senes.
\begin{proposition}
	[Uniqueness of morphisms via stalks]
	Consider a map of ringed spaces
	$(\pi, \pi^\sharp) \colon (X, \OO_X) \to (Y, \OO_Y)$
	and the corresponding map $\pi^\sharp_p$ of stalks.
	Then $\pi^\sharp$ is uniquely determined by $\pi^\sharp_p$.
\end{proposition}
\begin{proof}
	Given a section $s \in \OO_Y(U)$,
	let
	\[ t = \pi^\sharp_U(s) \in \OO_X(\pi\pre(U)) \]
	denote the image under $\pi^\sharp$.

	We know $t_p$ for each $p \in \pi\pre(U)$,
	since it equals $\pi^\sharp_p(t)$ by definition.
	That is, we know all the germs of $t$.
	So we know $t$.
\end{proof}

However, it seems clear that not every choice of
stalk morphisms will lead to $\pi^\sharp_U$:
some sort of ``continuity'' or ``compatibility'' is needed.
You can actually write down the explicit statement:
each sequence of compatible germs over $U$
should get mapped to a sequence of compatible germs over $\pi\pre(U)$.
We avoid putting up a formal statement of this for now,
because the statement is clumsy,
and you're about to see it in practice (where it will make more sense).

\begin{remark}
	[Isomorphisms are determined by stalks]
	One fact worth mentioning, that we won't prove, but good to know:
	a map of ringed spaces
	$(\pi, \pi^\sharp) \colon (X, \OO_X) \to (Y, \OO_Y)$
	is an isomorphism if and only if $\pi$ is a homeomorphism,
	and moreover $\pi^\sharp_p$ is an isomorphism for each $p \in X$.
\end{remark}

\section{Morphisms of locally ringed spaces}
On the other hand, we've seen that our stalks are local rings,
which enable us to actually talk about \emph{values}.
And so we want to add one more compatibility condition
to ensure that our notion of value is preserved.
Now the stalks at $p$ and $q$ in the previous picture might be different,
so $\kappa(p)$ and $\kappa(q)$ might even be different fields.
\begin{definition}
	A \vocab{morphism of locally ringed spaces}
	is a morphism of ringed spaces $\pi \colon X \to Y$
	with the following additional property:
	whenever $\pi(p) = q$,
	the map at the stalks also induces a well-defined ring homomorphism
	\[ \pi^\sharp_p \colon \kappa(q) \to \kappa(p). \]
\end{definition}
So we require $\pi^\sharp_p$ induces a field homomorphisms\footnote{Which
means it is automatically injective, by \Cref{prob:field_hom}.}
on the \emph{residue fields}.
In particular, since $\pi^\sharp(0) = 0$,
this means something very important:
\begin{moral}
	In a morphism of locally ringed spaces,
	a germ vanishes at $q$ if and only if
	the corresponding germ vanishes at $p$.
\end{moral}
\begin{exercise}
	[So-called ``local ring homomorphism'']
	Show that this is equivalent to requiring
	\[ (\pi^\sharp_p)\im(\km_{Y,q}) \subseteq \km_{X,p} \]
	or in English, a germ at $q$ has value zero
	iff the corresponding germ at $p$ has value zero.
\end{exercise}
I don't like this formulation
$(\pi^\sharp)\im(\km_{Y,q}) \subseteq \km_{X,p}$
as much since it hides the geometric intuition behind a lot of symbols:
that we want the notion of ``value at a point''
to be preserved in some way.

At this point, we can state the definition of a scheme,
and we do so, although we won't really use it for a few more sections.
\begin{definition}
	A \vocab{scheme} is a locally ringed space
	for which every point has an open neighborhood
	isomorphic to an affine scheme.
	A morphism of schemes is just a morphism of locally ringed spaces.
\end{definition}
In particular, $\Spec A$ is a scheme
(the open neighborhood being the entire space!).
And so let's start by looking at those.

\section{A few examples of morphisms between affine schemes}
Okay, sorry for lack of examples in previous few sections.
Let's make amends now,
where you can see all the moving parts in action.

\subsection{One-point schemes}
\begin{example}
	[$\Spec \RR$ is well-behaved]
	There is only one map $X = \Spec \RR \to \Spec \RR = Y$.
	Indeed, these are spaces with one point,
	and specifying the map $\RR = \OO_Y(Y) \to \OO_X(X) = \RR$
	can only be done in one way,
	since there is only one field automorphism of $\RR$ (the identity).
\end{example}
\begin{example}
	[$\Spec \CC$ horror story]
	There are multiple maps $X = \Spec \CC \to \Spec \CC = Y$,
	horribly enough!
	Indeed, these are spaces with one point,
	so again we're just reduced to specifying
	a map $\CC = \OO_Y(Y) \to \OO_X(X) = \CC$.
	However, in addition to the identity map,
	complex conjugation also works,
	as well as some so-called ``wild automorphisms'' of $\CC$.
\end{example}

This behavior is obviously terrible,
so for illustration reasons,
some of the examples use $\RR$ instead of $\CC$
to avoid the horror story we just saw.
However, there is an easy fix using ``scheme over $\CC$''
which will force the ring homomorphisms to fix $\CC$, later.

\begin{example}
	[$\Spec k$ and $\Spec k'$]
	In general, if $k$ and $k'$ are fields,
	we see that maps $\Spec k \to \Spec k'$
	are in bijection with field homomorphism $k' \to k$,
	since that's all there is left to specify.
\end{example}

\subsection{Examples of constant maps}
\begin{example}
	[Constant map to $(y-3)$]
	We analyze scheme morphisms
	\[ X = \Spec \RR[x] \taking{\pi} \Spec \RR[y] = Y \]
	which send all points of $X$ to $\km = (y-3) \in Y$.
	Constant maps are continuous no matter how bizarre your topology is,
	so this lets us just focus our attention on the sections.

	This example is simple enough that we can even do it by sections,
	as much as I think stalks are simpler.
	Let $U$ be any open subset of $Y$, then we need to specify a map
	\[ \pi^\sharp_U \colon \OO_Y(U) \to \OO_X(\pi\pre(U)). \]
	If $U$ does not contain $(y-3)$, then $\pi\pre(U) = \varnothing$,
	so $\OO_X(\varnothing) = 0$ is the zero ring and there is nothing to do.

	Conversely, if $U$ does contain $(y-3)$ then $\pi\pre(U) = X$,
	so this time we want to specify a map
	\[ \pi^\sharp_U \colon \OO_Y(U) \to \OO_X(X) = \RR[x] \]
	which satisfies restriction maps.
	Note that for any $U$, the element $y$ must map to a unit in $\RR[x]$;
	since $1/y$ is a section too for a subset of $U$ not containing $(y)$.
	Actually for any real number $c \ne 3$,
	$y-c$ must be map to a unit in $\RR[x]$.
	This can only happen if $y \mapsto 3 \in \RR[x]$.

	As we have specified $\RR[y] \mapsto \RR[x]$ with $y \mapsto 3$,
	that determines all the ring homomorphisms we needed.
\end{example}
But we could have used stalks, too.
We wanted to specify a morphism
\[ \RR[y]_{(y-3)} = \OO_{\Spec Y, (y-3)} \to \OO_{\Spec X, \kp} \]
for every prime ideal $\kp$,
sending compatible germs to compatible germs\dots
but wait, $(y-3)$ is spitting out all the germs.
So every \emph{individual} germ in $\OO_{\Spec Y, (y-3)}$
needs to yield a (compatible) germ above every point of $\Spec X$,
which is the data of an entire global section.
So we're actually trying to specify
\[ \RR[y]_{(y-3)} = \OO_{\Spec Y, (y-3)} \to \OO_{\Spec X}(\Spec X) = \RR[x]. \]
This requires $y \mapsto 3$, as we saw,
since $y-c$ is a unit of $\RR[x]$ for any $c \ne 3$.

\begin{example}
	[Constant map to $(y^2+1)$ does not exist]
	Let's see if there are constant maps
	$X = \Spec \RR[x] \to \Spec \RR[y] = Y$
	which send everything to $(y^2+1)$.
	Copying the previous example, we see that we want
	\[ \OO_Y(U) \to \OO_X(X) = \RR[x]. \]
	We find that $y$ and $1/y$ has nowhere to go:
	the same argument as last time shows that $y-c$ should be a unit of $\RR[x]$;
	this time for any real number $c$.

	Like this time, stalks show this too,
	even with just residue fields.
	We would for example need a field homomorphism
	\[ \CC = \kappa( (y^2+1) ) \to \kappa( (x) ) = \RR \]
	which does not exist.
\end{example}

You might already notice the following:
\begin{example}
	[The generic point repels smaller points]
	Changing the tune, consider maps $\Spec \CC[x] \to \Spec \CC[y]$.
	We claim that if $\km$ is a maximal ideal (closed point)
	of $\CC[x]$, then it can never be mapped to the generic point $(0)$ of $\CC[y]$.

	For otherwise, we would get a local ring homomorphism
	\[ \CC(y) \cong \OO_{\Spec k[y], (0)}
		\to \OO_{\Spec \CC[x], \km} \cong \CC[x]_\km \]
	which in particular means we have a map on the residue fields
	\[ \CC(y) \to \CC[x] / \km \cong \CC \]
	which is impossible, there is no such field homomorphism at all (why?).
\end{example}
The last example gives some nice intuition in general:
``more generic'' points tend to have bigger stalks
than ``less generic'' points, hence repel them.

\subsection{The map $t \mapsto t^2$}
We now consider what we would think
of as the map $t \mapsto t^2$.
\begin{example}
	[The map $t \mapsto t^2$]
	We consider a map
	\[ \pi \colon X = \Spec \CC[x] \to \Spec \CC[y] = Y \]
	defined on points as follows:
	\begin{align*}
		\pi\left( (0) \right) &= (0) \\
		\pi\left( (x-a)  \right) &= (y-a^2).
	\end{align*}
	You may check if you wish this map is continuous.
	I claim that, surprisingly,
	you can actually read off $\pi^\sharp$ from just this
	behavior at points.
	The reason is that we imposed the requirement
	that a section $s$ can vanish at $\kq \in Y$
	if and only if $\pi^\sharp_X(s)$ vanishes at $\kp \in X$,
	where $\pi(\kp) = \kq$.
	So, now:
	\begin{itemize}
		\ii Consider the section $y \in \OO_Y(Y)$,
		which vanishes only at $(y) \in \Spec \CC[y]$;
		then its image $\pi^\sharp_Y(y) \in \OO_X(X)$
		must vanish at exactly $(x) \in \Spec \CC[x]$,
		so $\pi^\sharp_Y(y) = x^n$ for some integer $n \ge 1$.
		\ii Consider the section $y-4 \in \OO_Y(Y)$,
		which vanishes only at $(y-4) \in \Spec \CC[y]$;
		then its image $\pi^\sharp_Y(y-4) \in \OO_X(X)$
		must vanish at exactly $(x-2) \in \Spec \CC[x]$
		and $(x+2) \in \Spec \CC[x]$.
		So $\pi^\sharp_Y(y)-4$ is divisible by $(x-2)^a(x+2)^b$
		for some $a \ge 1$ and $b \ge 1$.
	\end{itemize}
	Thus $y \mapsto x^2$ in the top level map of sections $\pi^\sharp_Y$:
	and hence also in all the maps of sections
	(as well as at all the stalks).
\end{example}
The above example works equally well if $t^2$ is replaced
by some polynomial $f(t)$,
so that $(x-a)$ maps to $(y-f(y))$.
The image of $y$ must be a polynomial $g(x)$
with the property that $g(x)-c$ has the same roots as $f(x)-c$
for any $c \in \CC$.
Put another way, $f$ and $g$ have the same values, so $f = g$.

\begin{remark}
	[Generic point stalk overpowered]
	I want to also point out that you can read off the polynomial
	just from the stalk at the generic point:
	for example, the previous example has
	\[ \CC(y) \cong \OO_{\Spec \CC[y], (0)}
		\to \OO_{\Spec \CC[x], (0)} \cong \CC(x)
		\qquad y \mapsto x^2. \]
	This is part of the reason why generic points are so powerful.
	We expect that with polynomials,
	if you know what happens to a ``generic'' point,
	you can figure out the entire map.
	This intuition is true:
	knowing where each germ at the generic point goes
	is enough to tell us the whole map.
\end{remark}

\subsection{An arithmetic example}
\begin{example}
	[{$\Spec \ZZ[i] \to \Spec \ZZ$}]
	We now construct a morphism of schemes
	$\pi \colon \Spec \ZZ[i] \to \Spec \ZZ$.
	On points it behaves by
	\begin{align*}
		\pi\left( (0) \right) &= (0) \\
		\pi\left( (p) \right) &= (p) \\
		\pi\left( (a+bi) \right) &= (a^2+b^2)
	\end{align*}
	where $a+bi$ is a Gaussian prime:
	so for example $\pi( (2+i) ) = (5)$ and $\pi( (1+i) ) = (2)$.
	We could figure out the induced map on stalks now,
	much like before, but in a moment we'll have a big theorem
	that spares us the trouble.
\end{example}


\section{The big theorem}
We did a few examples of $\Spec A \to \Spec B$ by hand,
specifying the full data of a map of locally ringed spaces.
It turns out that in fact, we didn't to specify that much data,
and much of the process can be automated:
\begin{proposition}
	[Affine reconstruction]
	\label{prop:affine_reconstruction}
	Let $\pi \colon \Spec A \to \Spec B$ be a map of schemes.
	Let $\psi \colon B \to A$ be the ring homomorphism
	obtained by taking global sections, i.e.
	\[ \psi= \pi^\sharp_B \colon \OO_{\Spec B}(\Spec B)
		\to \OO_{\Spec A}(\Spec A). \]
	Then we can recover $\pi$ given only $\psi$;
	in fact, $\pi$ is given explicitly by
	\[ \pi(\kp) = \psi\pre(\kp) \]
	and
	\[ \pi^\sharp_\kp \colon \OO_{Y,\pi(\kp)} \to \OO_{X,\kp}
			\quad\text{ by }\quad f/g \mapsto \psi(f)/\psi(g).  \]
\end{proposition}
This is the big miracle of affine schemes.
Despite the enormous amount of data packaged into the definition,
we can compress maps between affine schemes
into just the single ring homomorphism on the top level.
\begin{proof}
	This requires two parts.
	\begin{itemize}
		\ii We need to check that the maps agree on \emph{points};
		surprisingly this is the harder half.
		To see how this works, let $\kq = \pi(\kp)$.
		The key fact is that a function $f \in B$ vanishes on $\kq$
		if and only if $\pi^\sharp_B(f)$ vanishes on $\kp$
		(because $\pi^\sharp_B$ is supposed to be a
		homomorphism of \emph{local} rings).
		Therefore,
		\begin{align*}
			\pi(\kp) &= \kq = \left\{ f \in B \mid f \in \kq \right\} \\
			&= \left\{ f \in B \mid f \text{ vanishes on } \kq \right\} \\
			&= \left\{ f \in B \mid \pi^\sharp_B(f) \text{ vanishes on } \kp \right\} \\
			&= \left\{ f \in B \mid \pi^\sharp_B(f) \in \kp \right\}
			= \left\{ f \in B \mid \psi(f) \in \kp \right\} \\
			&= \psi\pre(\kp).
		\end{align*}

		\ii We also want to check the maps on the stalks is the same.
		Suppose $\kp \in \Spec A$, $\kq \in \Spec B$,
		and $\kp \mapsto \kq$ (under both of the above).

		In our original $\pi$, consider the map
		$\pi^\sharp_\kp \colon B_\kq \to A_\kp$.
		We know that it sends each $f \in B$ to $\psi(f) \in A$,
		by taking the germ of each global section $f \in B$ at $\kq$.
		Thus it must send $f/g$ to $\psi(f) / \psi(g)$,
		being a ring homomorphism, as needed. \qedhere
	\end{itemize}
\end{proof}

All of this suggests a great idea:
if $\psi \colon B \to A$ is \emph{any} ring homomorphism,
we ought to be able to construct a map of schemes
by using fragments of the proof we just found.
The only extra work we have to do is verify that
we get a continuous map in the Zariski topology,
and that we can get a suitable $\pi^\sharp$.

We thus get the huge important theorem about affine schemes.
\begin{theorem}
	[$\Spec A \to \Spec B$ is just $B \to A$]
	These two construction gives a bijection between
	ring homomorphisms $B \to A$ and $\Spec A \to \Spec B$.
\end{theorem}
\begin{proof}
	We have seen how to take each
	$\pi \colon \Spec A \to \Spec B$
	and get a ring homomorphism $\psi$.
	\Cref{prop:affine_reconstruction} shows this map is injective.
	So we just need to check it is surjective --- that
	every $\psi$ arises from some $\pi$.

	Given $\psi \colon B \to A$, we define
	$(\pi, \pi^\sharp) \colon \Spec A \to \Spec B$
	by copying \Cref{prop:affine_reconstruction}
	and checking that everything is well-definced.
	The details are:
	\begin{itemize}
	\ii For each prime ideal $\kp \in \Spec A$,
	we let $\pi(\kp) = \psi\pre(\kp) \in \Spec B$
	(which by \Cref{prob:prime_preimage} is also prime).
	\begin{exercise}
		Show that the resulting map $\pi$ is continuous
		in the Zariski topology.
	\end{exercise}
	\ii Now we want to also define maps on the stalks,
	and so for ecah $\pi(\kp) = \kq$ we set
	\[ B_\kq \ni \frac fg \mapsto \frac{\psi(f)}{\psi(g)} \in A_\kp. \]
	This makes sense since $g \notin \kq \implies \psi(g) \notin \kp$.
	Also $f \in \kq \implies \psi(f) \in \kp$,
	we find this really is a local ring homomorphism
	(sending the maximal ideal of $B_\kq$ into the one of $A_\kp$).

	Observe that if $f/g$ is a \emph{section}
	over an open set $U \subseteq B$
	(meaning $g$ does not vanish at the primes in $U$),
	then $\psi(f)/\psi(g)$ is a section over $\pi\pre(U)$
	(meaning $\psi(g)$ does not vanish at the primes in $\pi\pre(U)$).
	Therefore, compatible germs over $B$
	get sent to compatible germs over $A$, as needed.
	\end{itemize}
	Finally, the resulting $\pi$ has $\pi^\sharp_B = \psi$ on global sections,
	completing the proof.
\end{proof}

This can be summarized succinctly using category theory:
\begin{corollary}
	[Categorical interpretation]
	The opposite category of rings $\catname{CRing}\op$,
	is ``equivalent'' to the category of affine schemes, $\catname{AffSch}$,
	with $\Spec$ as a functor.
\end{corollary}
This means for example that $\Spec A \cong \Spec B$,
naturally, whenever $A \cong B$.

To make sure you realize that this theorem is important,
here is an amusing comment I found on MathOverflow
while reading about algebraic geometry
references\footnote{From \url{https://mathoverflow.net/q/2446/70654}.}:
\begin{quote}
	\small\sffamily\itshape
	He [Hartshorne] never mentions that the category of affine schemes
	is dual to the category of rings, as far as I can see.
	I'd expect to see that in huge letters
	near the definition of scheme.
	How could you miss that out?
\end{quote}

\section{More examples of scheme morphisms}
Now that we have the big hammer,
we can talk about examples much more briefly
than we did a few sections ago.
Before throwing things around,
I want to give another definition
that will eliminate the weird behavior we saw with
$\CC \to \CC$ having nontrivial field automorphisms:
\begin{definition}
	Let $S$ be a scheme.
	A \vocab{scheme over $S$} or \vocab{$S$-scheme} is a scheme $X$
	together with a map $X \to S$.
	A morphism of $S$-schemes is a scheme morphism $X \to Y$
	such that the diagram
	\begin{tikzcd}
		X \ar[r] \ar[rd] & Y \ar[d] \\ & S
	\end{tikzcd}
	commutes.
	Often, if $S = \Spec k$, we will use refer
	to schemes over $k$ or $k$-schemes for short.
\end{definition}
\begin{example}
	[{$\Spec k[\dots]$}]
	If $X = \Spec k[x_1, \dots, x_n] / I$ for some ideal $I$,
	then $X$ being a $k$-scheme in a natural way;
	since we have an obvious homomorphism
	$k \injto k[x_1, \dots, x_n] / I$
	which gives a map $X \to \Spec k$.
\end{example}

\begin{example}
	[{$\Spec \CC[x] \to \Spec \CC[y]$}]
	As $\CC$-schemes, maps $\Spec \CC[x] \to \Spec \CC[y]$
	coincide with ring homomorphisms
	from $\psi \colon \CC[y] \to \CC[x]$ such that the diagram
	\begin{center}
	\begin{tikzcd}
		\CC[x] & \CC[y] \ar[l, "\psi"'] \\
		& \ar[lu, hook] \ar[u, hook] \CC
	\end{tikzcd}
	\end{center}
	We see that the ``over $\CC$'' condition is eliminating
	the pathology from before:
	the $\psi$ is required to preserve $\CC$.
	So the morphism is determined by the image of $y$,
	i.e.\ the choice of a polynomial in $\CC[x]$.
	For example, if $\psi(y) = x^2$
	we recover the first example we saw.
	This matches our intuition that these maps should correspond
	to polynomials.
\end{example}

\begin{example}
	[$\Spec \OO_K$]
	This generalize $\Spec \ZZ[i]$ from before.
	If $K$ is a number field and $\OO_K$,
	then there is a natural morphism $\Spec \OO_K \to \Spec \ZZ$
	from the (unique) ring homomorphism $\ZZ \injto \OO_K$.
	Above each rational prime $(p) \in \ZZ$,
	one obtains the prime ideals that $p$ splits as.
	(We don't have a way of capturing ramification yet, alas.)
\end{example}

\section{A little bit on non-affine schemes}
We can finally state the isomorphism that we wanted for a long time
(first mentioned in \Cref{subsec:distinguished_open_affine}):
\begin{theorem}
	[Distinguished open sets are isomorphic to affine schemes]
	Let $A$ be a ring and $f$ an element.
	Then
	\[ \Spec A[1/f] \cong D(f) \subseteq \Spec A. \]
\end{theorem}
\begin{proof}
	Annoying check, not included yet.
	(We have already seen the bijection of prime ideals, at the level of points.)
\end{proof}

\begin{corollary}
	[Open subsets are schemes]
	\listhack
	\begin{enumerate}[(a)]
		\ii Any nonempty open subset of an affine scheme is itself a scheme.
		\ii Any nonempty open subset of any scheme (affine or not) is itself a scheme.
	\end{enumerate}
\end{corollary}
\begin{proof}
	Part (a) has essentially been done already:
	\begin{ques}
		Combine \Cref{thm:distinguished_base}
		with the previous proposition to deduce (a).
	\end{ques}
	Part (b) then follows by noting that if $U$ is an open set,
	and $p$ is a point in $U$,
	then we can take an affine open neighborhood $\Spec A$ at $p$,
	and then cover $U \cap \Spec A$ with distinguished
	open subsets of $\Spec A$ as in (a).
\end{proof}

We now reprise \Cref{subsec:punctured_plane}
(except $\CC$ will be replaced by $k$).
We have seen it is an open subset $U$ of $\Spec k[x,y]$, so it is a scheme.
\begin{ques}
	Show that in fact $U$ can be covered by
	two open sets which are both affine.
\end{ques}
However, we show now that you really do need two open sets.
\begin{proposition}
	[Famous example: punctured plane isn't affine]
	The punctured plane $U = (U, \OO_U)$,
	obtained by deleting $(x,y)$ from $\Spec k[x,y]$,
	is not isomorphic to any affine scheme $\Spec B$.
\end{proposition}

The intuition is that $\OO_U(U) = k[x,y]$, but $U$ is not the plane.
\begin{proof}
	We already know $\OO_U(U) = k[x,y]$
	and we have a good handle on it.
	For example, $y \in \OO_U(U)$ is a global section
	which vanishes on what looks like the $y$-axis.
	Similarly, $x \in \OO_X(X)$ is a global section
	which vanishes on what looks like the $y$-axis.
	In particular, no point of $U$ vanishes at both.

	Now assume for contradiction that we have an isomorphism
	\[ \psi \colon \Spec B \to U. \]
	By taking the map on global sections (part of the definition),
	\[ k[x,y] = \OO_U(U) \taking{\psi^\sharp}
		\OO_{\Spec B}(\Spec B) \cong B. \]
	The global sections $x$ and $y$ in $\OO_U(U)$
	should then have images $a$ and $b$ in $B$;
	and it follows we have a ring isomorphism $B \cong k[a,b]$.

	\begin{center}
	\begin{asy}
		unitsize(0.9cm);
		draw( (0,-3)--(0,3), red );
		draw( (-3,0)--(3,0), red );
		fill(box( (-3,-3), (3,3) ), opacity(0.2)+lightcyan);
		opendot(origin, blue+1.5);
		label("$\mathcal V(x)$", (3,0), dir(135), red);
		label("$\mathcal V(y)$", (0,3), dir(-45), red);
		label("$U$", (0,3), dir(90), deepcyan);
		label("$\mathcal O_U(U) = k[x,y]$", (0,-3), dir(-90), deepcyan);
		add(shift(-8,0)*CC());
		fill(box( (-3,-3), (3,3) ), opacity(0.2)+lightblue);
		path a = (-3,-1)..(-1,0.7)..(0.2,0.5)..(2,1)..(3,0.8);
		path b = (-0.5,-3)..(0.4,-1)..(0.2,0.5)..(-0.4,2)..(-0.7,3);
		draw(a, red);
		draw(b, red);
		label("$\mathcal V(a)$", (3,0.8), dir(225), red);
		label("$\mathcal V(b)$", (-0.7,3), dir(225), red);
		dot("$(a,b)$", (0.2,0.5), dir(225), red);
		label("$\operatorname{Spec} B$", (0,3), dir(90), blue);
		label("$\mathcal O_{\operatorname{Spec} B}(\operatorname{Spec} B) \cong k[a,b]$", (0,-3), dir(-90), blue);
	\end{asy}
	\end{center}

	Now in $\Spec B$, $\VV(a) \cap \VV(b)$ is a closed
	set containing a single point, the maximal ideal $\km = (a,b)$.
	Thus in $\Spec B$ there is exactly
	one point vanishing at both $a$ and $b$.
	\emph{Because we required morphisms of schemes to preserve values}
	(hence the big fuss about locally ringed spaces),
	that means there should be a single point of $U$
	vanishing at both $x$ and $y$.
	But there isn't --- it was the origin we deleted.
\end{proof}

\section{Where to go from here}
% \todo{I think I want to write more, but in a later edition}
This chapter concludes the long setup for the definition of a scheme.
For now, this unfortunately is as far as I have time to go.
So, if you want to actually see how schemes are used in ``real life'',
you'll have to turn elsewhere.

A good reference I happily recommend is \cite{ref:gathmann};
a more difficult (and famous) one is \cite{ref:vakil}.
See \Cref{ch:refs} for further remarks.

\section{\problemhead}
\begin{problem}
	Given an affine scheme $X = \Spec R$,
	show that there is a unique morphism of schemes $X \to \Spec \ZZ$,
	and describe where it sends points of $X$.
	\begin{hint}
		Use the fact that $\catname{AffSch} \simeq \catname{CRing}$.
	\end{hint}
	\begin{sol}
		Since $\ZZ$ is the initial object of $\catname{CRing}$,
		it follows $\Spec \ZZ$ is the final object of $\catname{AffSch}$.
		$\kp$ gets sent to the characteristic of the field $\OO_{X,\kp} / \km_{X,\kp}$.
	\end{sol}
\end{problem}

\begin{problem}
	Is the open subset of $\Spec \ZZ[x]$
	obtained by deleting the point $\km = (2,x)$
	isomorphic to some affine scheme?
\end{problem}

\endinput

\section{Projective scheme}
\prototype{Projective varieties, in the same way.}
The most important class of schemes which are not affine are
\emph{projective} schemes.
The complete the obvious analogy:
\[
	\frac{\text{Affine variety}}{\text{Projective variety}}
	=
	\frac{\text{Affine scheme}}{\text{Projective scheme}}.
\]
Let $S$ be \emph{any} (commutative) graded ring, like $\CC[x_0, \dots, x_n]$.
\begin{definition}
	We define $\Proj S$, the \vocab{projective scheme over $S$}:
	\begin{itemize}
		\ii As a set, $\Proj S$ consists of \emph{homogeneous prime ideals}
		$\kp$ which do not contain $S^+$.
		\ii If $I \subseteq S$ is homogeneous, then
		we let $\Vp(I) = \{ \kp \in \Proj S \mid I \subseteq \kp \}$.
		Then the \vocab{Zariski topology} is imposed by declaring
		sets of the form $\Vp(I)$ to be closed.
		\ii We now define a pre-sheaf $\SF$ on $\Proj S$ by
		\[ \SF(U) =
			\left\{ \frac{f}{g} \mid
			g(\kp) \neq 0 \; \forall \kp \in U \text{ and }
			\deg f = \deg g \right\}.
		\]
		In other words, the rational functions are quotients $f/g$
		where $f$ and $g$ are \emph{homogeneous of the same degree}.
		Then we let \[ \OO_{\Proj S} = \SF\sh \] be the sheafification.
	\end{itemize}
\end{definition}
\begin{definition}
	The \vocab{distinguished open sets} $D(f)$ of the $\Proj S$
	are defined as $\left\{ \kp \in \Proj S : f(\kp) \neq 0 \right\}$,
	as before; these form a basis for the Zariski topology of $\Proj S$.
\end{definition}
Now, we want analogous results as we did for affine structure sheaf.
So, we define a slightly modified localization:
\begin{definition}
	Let $S$ be a graded ring.
	\begin{enumerate}[(i)]
		\ii For a prime ideal $\kp$, let
		\[ S_{(\kp)} = \left\{ \frac fg \mid g(\kp) \neq 0 \text{ and }
			\deg f = \deg g \right\} \]
		denote the elements of $S_\kp$ with ``degree zero''.
		\ii For any homogeneous $g \in S$ of degree $d$, let
		\[ S_{(g)} = \left\{ \frac{f}{g^r} \mid
			\deg f = r \deg g \right\} \]
		denote the elements of $S_g$ with ``degree zero''.
	\end{enumerate}
\end{definition}

\begin{theorem}
	[On the projective structure sheaf]
	Let $S$ be a graded ring and let $\Proj S$ the associated projective scheme.
	\begin{enumerate}[(a)]
		\ii Let $\kp \in \Proj S$.
		Then $\OO_{\Proj S, \kp} \cong S_{(\kp)}$.
		\ii Suppose $g$ is homogeneous with $\deg g > 0$. Then
		\[ D(g) \cong \Spec S_{(g)} \]
		as locally ringed spaces.
		In particular, $\OO_{\Proj S}(D(g)) \cong S_{(g)}$.
	\end{enumerate}
\end{theorem}
\begin{ques}
	Conclude that $\Proj S$ is a scheme.
\end{ques}

Of course, the archetypal example is that
\[ \Proj \CC[x_0, x_1, \dots, x_n] / I \]
corresponds to the projective subvariety of $\CP^n$
cut out by $I$ (when $I$ is radical).
In the general case of an arbitrary ideal $I$, we
call such schemes \vocab{projective subscheme} of $\CP^n$.
For example, the ``double point'' in  $\CP^1$
is given by $\Proj[x_0,x_1]/(x_0^2)$.

\begin{remark}
	No comment yet on what the global sections of $\OO_{\Proj S}(\Proj S)$ are.
	(The theorem above requires $\deg g > 0$, so we cannot just take $g=1$.)
	One might hope that in general $\OO_{\Proj S}(\Proj S) \cong S^0$
	in analogy to our complex projective varieties, but
	one needs some additional assumptions on $S$ for this to hold.
\end{remark}



\section{Morphisms of sheaves}
First, recall that a sheaf is a contravariant functor (pre-sheaf)
with extra conditions. In light of this, it is not hard to guess
the definition of a morphism of pre-sheaves:
\begin{definition}
	A \vocab{morphism of (pre-)sheaves} $\alpha : \SF \to \SG$ on the same
	space $X$ is a \textbf{natural transformation} of the underlying functors.
	Isomorphism of sheaves is defined in the usual way.
\end{definition}
\begin{ques}
	Show that this amounts to: for each $U \subseteq X$ we need to specify
	a morphism $\alpha_U : \SF(U) \to \SG(U)$ such that the diagram
	\begin{center}
	\begin{tikzcd}
		\SF(U_2) \ar[r, "\alpha_{U_2}"]
			\ar[d, "\res_{U_1, U_2}"']
			& \SG(U_2) \ar[d, "\res_{U_1, U_2}"]\\
		\SF(U_1) \ar[r, "\alpha_{U_1}"'] & \SG(U_1)
	\end{tikzcd}
	\end{center}
	commutes any time that $U_1 \subseteq U_2$.
\end{ques}
However, in the sheaf case we like stalks more than sections because
they are theoretically easier to think about.
And in fact:
\begin{proposition}
	[Morphisms determined by stalks]
	A morphism of sheaves $\alpha : \SF \to \SG$ induces a morphism of stalks
	\[ \alpha_p : \SF_p \to \SG_p \]
	for every point $p \in X$.
	Moreover, the sequence $(\alpha_p)_{p \in X}$ determines $\alpha$ uniquely.
\end{proposition}
\begin{proof}
	The morphism $\alpha_p$ itself is just
	$(s, U) \xmapsto{\alpha_p} (\alpha_U(s), U)$.
	\begin{ques}
		Show this is well-defined.
	\end{ques}
	Now suppose $\alpha , \beta : \SF \to \SG$ satisfy $\alpha_p = \beta_p$
	for every $p$. We want to show $\alpha_U(s) = \beta_U(s)$
	for every $s \in \SF(U)$.
	\begin{ques}
		Verify this using the description of sections
		as sequences of germs. \qedhere
	\end{ques}
\end{proof}
Thus a morphism of sheaves can be instead modelled as a morphism
of all the stalks. We will see later on that this viewpoint is quite useful.
